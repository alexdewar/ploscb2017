\section*{Abstract}
[PRG NEW TITLE SUGGESTION Neural coding in \emph{Drosophila}: how do small population codes relate to visually guided behaviours?]
All organisms must be able to respond adaptively to a complex, changing world.
Yet the computational power available is constrained by biology and evolution, favouring mechanisms that are parsimonious yet robust.
Here we investigate the information carried in small populations of visually responsive neurons in \emph{Drosophila melanogaster}.
These small populations of so-called `ring neurons', found in the ellipsoid body of the central complex, are key for complex visual tasks.
Recently the receptive fields of these neurons have been mapped, allowing us to take a bottom-up approach in investigating what behaviours would be supported by these small population codes.
Previously, these neurons have been implicated in behaviours such as pattern recognition and in a simulation of the classic pattern recognition experiments, we show that the population code from a small set of ring neurons matches observed fly behaviour.
However, performance (of the population code and the fly) is not perfect.
The pattern discrimination of the population of simulated ring neurons can be easily improved by the addition of extra neurons.
Such an straightforward adaptation would surely have evolved if pattern discrimination was important for the fly.
Using artificial neural networks to ask how easy it is to decode stimulus information from the population code provided by ring neurons we showed that these small populations are ideal for encoding information about the size, position and orientation of objects, which are more relevant behavioural parameters for a fly than abstract pattern properties.
In this instance, computational modelling is key in investigating how perceptual circuits put information at the service of behaviour rather than how they preserve visual information.