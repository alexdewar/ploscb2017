\begin{figure}
\centering
\includegraphics{figures/recap}
\caption{PRG 31/7/15 Simulations of classic paradigms for the study of fly visually guided behaviour. A: In Buridan's paradigm \protect\cite{Bulthoff1982,Gotz1980} a fly is placed in an arena between two large vertical bars, it will walk back and forth until exhaustion. In a simulation of such an arena we can look at the output of ring neurons for simulated flies at different positions. B and C: The mean magnitude of the output of R2 (blue) and R4d (green) filters, for simulated flies with different headings relative to the bar. The labels in A represent locations relative to the bar giving the different bar sizes in B and C. D: The raw outputs of R2 and R4d filters can be used to drive orientation towards a bar stimulus. Top: a simple proportional-integral-derivative controller (PID controller) for bar homing. Bottom: A vector field of orientations for a simulated fly driven by the simple PID controller.
E: A standard experimental paradigm for testing the pattern discrimination abilities of \emph{Drosophila} \protect\cite{Ernst1999}. The fly is held tethered in a drum. As the fly attempts to rotate about its yaw-axis, the drum rotates in the opposite direction, thus allowing the fly to centre a portion of the pattern in its view. By monitoring the fly's heading, one can surmise whether there is a spontaneous preference for one of the patterns. Whether the fly can learn to head towards one pattern is tested by adding a laser that punishes the fly for facing one of the patterns. Shown inside the drum are the visual \acp{RF} for one symmetric pair of ring neurons.
F and G: The \ac{rms} difference in R2(blue) and R4d (green) output as 2 sets of patterns are rotated. In each case the reference level for the \ac{rms} measure are the \ac{RF} outputs when the simulated fly is at 0\degree. If there is a large \ac{rms} activity difference for 0\degree\ \emph{vs} 90\degree\ then the two components of the pattern array are more easily discriminable using this particular encoding.
F, G: We show that there is a much smaller difference in output 0\degree\ \emph{vs} 90\degree\ when triangles are aligned about the vertical centre of mass (F) than not (G).
This mirrors real flies' performance on this task \protect\cite{Ernst1999}.}
\label{fig:recap}
\end{figure}
