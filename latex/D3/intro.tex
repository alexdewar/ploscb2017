\section*{Introduction}

Insects, like vertebrates, exhibit a great many behaviours key to survival, including place-homing \cite{Ofstad2011}, recognition of individual conspecifics \cite{Sheehan2008}, flight control \cite{Taylor2001,Gotz1987}, complex communication \cite{VonFrisch1967} and nest-building [e.g. termite stuff].
Moreover, these feats are carried out with nervous systems orders of magnitude smaller than their vertebrate cousins'.
This combination of behavioural complexity with (relatively) simple nervous systems and well-defined ecological niches makes many invertebrate species attractive as model organisms.

The question is then how best to study these animals that are so different from mammalian model organisms.
Should we look for homologues of vertebrate behaviours or instead try to understand how insect behaviours relate to specific, ecologically relevant tasks?
These approaches are often referred to as `top down' and `bottom up', respectively.
We feel there are a number of reasons to prefer the latter approach.
One reason is that keeping in mind an animal's specific ecology, sometimes it turns out that a task can more simply be solved with a task-specific heuristic rather than a more `general-purpose' cognitive tool.
For example, a male fiddler crab \emph{Uca pugilator} has two visual tasks that it must perform to survive and reproduce.
One is to avoid predators, the other is to attract females or repel other males, both of which are achieved by a claw-waving display.
As it turns out, these tasks can be carried out with a simple heuristic: treat anything above the horizon as a predator and anything below as a conspecific \cite{Layne1997}.
In this way, what at the outset one might think of as two `visual recognition' tasks can be greatly simplified, along with the neural machinery needed to support it.

Sometimes, this approach can shed new light on classical animal experiments.
For example, Wystrach and colleagues \cite{Wystrach2011} showed that ants (\emph{Gigantiops destructor}) make the same kinds of error as vertebrates on a task taken to indicate that animals code their spatial surroundings in terms of geometry.
An animal is trained to find food in one corner of a rectangular arena; when the animal is released during testing, it will head for either the correct corner or its diagonal opposite with equal likelihood.
Moreover, this holds true even when the two corners are disambiguated by the presence of additional visual features, and this is taken to indicate that there is a geometry module which takes precedence over other visual stimuli in a spatial task.
In the case of the ants, however, the data were better explained with a simple image-matching process (rotational image difference function \cite{Philippides2011,Zeil2003}) than by a model based on visual geometry \cite{Wystrach2011}.
Hence, the ants' performance can be better understood in terms of `view-based' homing \cite{Wystrach2013,Philippides2011,Baddeley2011,Lent2010}, than an abstract ability like `geometry recognition.'
Whilst it may be interesting to discover that an organism can perform a certain task, it is only through examining the underlying mechanism that the true significance of this will be apparent.
Otherwise it can be tempting to assume the ability is supported by a specific cognitive module \cite{Fodor1983} where none is needed.

Several classes of ring neuron in the \emph{Drosophila} ellipsoid body have been implicated in visual behaviours (R1: place homing \cite{Ofstad2011,Sitaraman2010,Sitaraman2008}; R2: pattern recognition \cite{Pan2009,Liu2006,Ernst1999}; R4: bar fixation memory \cite{Neuser2008}).
Recently, Seelig and Jayaraman \cite{Seelig2013} have described the form of the visual \acp{RF} for two subtypes of ring neuron---R2 and R4d---of which there are only 14 and 28, respectively.
In this paper, we intend to show that behavioural responses to experimental stimuli known to depend on these neurons are predicted on the basis of the information implicit in these neurons' responses, without the need for this information to be fed into specialised cognitive mechanisms.

