\section{Abstract}
All organisms wishing to survive and reproduce must be able to respond adaptively to a complex, changing world. Yet the computational power available is constrained by biology and evolution, favouring mechanisms that are parsimonious yet robust. Here we investigate the information carried in small populations of visually responsive neurons in \emph{Drosophila melanogaster}. These so-called `ring neurons', projecting to the ellipsoid body of the central complex, are required for complex visual tasks such as pattern recognition. Recently the receptive fields of these neurons have been mapped, allowing us to investigate how well they can support such behaviours. For instance, in a simulation of classic pattern discrimination experiments, we show that the patterns of output from the ring neurons matches observed fly behaviour. However, performance of the neurons (like flies) is not perfect and can be easily improved by addition of extra neurons, suggesting the neuron's receptive fields are not optimised for recognising abstract shapes, a conclusion which casts doubt on cognitive explanations of fly behaviour in pattern recognition assays. Using artificial neural networks, we then assess how easy it is to decode more general information about stimulus shape from the ring neuron population code. We show that these neurons are well-suited for encoding information about size, position and orientation, which are more relevant behavioural parameters for a fly than abstract pattern properties. This leads us to suggest that in order to understand the properties of neural systems, one must consider how perceptual circuits put information at the service of behaviour rather than how they preserve visual information.