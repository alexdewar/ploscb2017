
\section*{Materials and Methods}
\subsection{Turning visual receptive field data into visual filters}
\label{sec:methods:preprocessing}
The RFs used in these simulations were based on the data presented in\cite{Seelig2013}.
We first extract the image representations of the RFs from the figure (Extended Data Figure 8); this gives us images of $112\times 252$ pixels for R2 neurons and $88\times 198$ pixels for R4d.
Given the visual field is taken as $120\degree\times 270\degree$, this corresponds to a resolution of $0.871~\mathrm{deg}^2$ and $0.538~\mathrm{deg}^2$ per pixel, respectively.
As data is given for multiple flies, we averaged the RFs for the different glomeruli across flies ($2\le N(\mathrm{R2}) \le 6, 4\le N(\mathrm{R4})\le 7$). This process is summarised in Fig.~\ref{fig:avkernels}. Each point on the image was assigned a value ranging from --1 for maximum inhibition to 1 for maximum excitation, based on the values given by the colour scale bars in\cite{Seelig2013}.
These images were then thresholded to give a kernel $g(i,j)$:
$$
g(i,j) = \left\{ \begin{array}{rl}
		   1 & \mbox{for } R_{i,j} \ge T; \\
                  -1 & \mbox{for } R_{i,j} \le -T; \\
                   0 & \mbox{otherwise.}
                  \end{array}
          \right.
$$
where $g(i,j)$ is the ($i,j$)th pixel of the kernel, $R_{i,j}$ is the ($i,j$)th value of the processed receptive field image and $T$ is the threshold value, here $0.25$ (Fig.~\ref{fig:avkernels}A).

We took the centroid of the largest excitatory region as the `centre' of each of the kernels.
The excitatory regions were then extracted using Matlab's \texttt{bwlabeln} function (with eight-connectivity) and the centroid, $(x,y)$, with the \texttt{regionprops} function.
The mean centroid, $(\bar{x},\bar{y})$, across flies is then calculated and the kernels are recentred on this point:
$$
\hat{g}(i,j) = \left\{ \begin{array}{ll} g(i+y-\bar{y},j+x-\bar{x}) & \mbox{for } 1\le i+y-\bar{y}\le m \mbox{ and } 1\le j+x-\bar{x}\le n;\\
0 & \mbox{otherwise.} \end{array} \right.
$$
where $\hat{g}(i,j)$ is the recentred kernel (Fig.~\ref{fig:avkernels}C).

We next calculate the average kernel across flies, $\bar{g}(i,j)$, and threshold again:
\begin{align*}
\bar{g}(i,j) &= \left\{ \begin{array}{rl}
			-1 & \mbox{for } c \le -T; \\
			 1 & \mbox{for } c \ge T; \\
			 0 & \mbox{otherwise.} 
			\end{array} \right. \\
\mbox{where } c &= \frac{1}{|\mathbf{G}|}\sum\limits_{\hat{g} \in \mathbf{G}} \hat{g}(i,j)
\end{align*}
where $\mathbf{G}$ is the set of kernels being averaged and $T$ is the threshold (again: 0.25). Note that instead of thresholding then averaging the raw images, $R$, before thresholding them again, we could have averaged the raw pixel values. The reason we did not do so was to reduce noise on the raw images; tests showed a negligible difference in performance when doing the latter.

In order to calculate the activation for a given RF on presentation of an image the RF must first be resized to the same size as the image.
This is accomplished by resizing the average RF, $\bar{g}(i,j)$ (using Matlab's \texttt{imresize} function with appropriate normalisation).
Finally, the kernel is rethresholded and the excitatory and inhibitory regions are assigned different values:
$$
K_{i,j} = \left\{
\begin{array}{rl}
\frac{1}{N_\mathrm{exc}}, & \mbox{for } \bar{g}(i,j) = 1; \\
\frac{1}{N_\mathrm{inh}}, & \mbox{for } \bar{g}(i,j) = -1; \\
0, & \mbox{otherwise.}
\end{array}
\right.
$$
where $N_\mathrm{exc}$ and $N_\mathrm{inh}$ indicate the number of excitatory and inhibitory pixels, respectively.
This method of allocating values has the result that the activation (see below) for an all-white or -black image will be zero.

The activation of an average kernel, $K$, to the presentation of a greyscale image, $I$, at rotation $\theta$, is then:
\begin{equation}
\label{eq:act}
\begin{array}{rl}
A(I,K,\theta) = {\sum\limits^m_{i=1} \sum\limits^n_{j=1} I_{i,j}(\theta)K_{i,j}}, &\mathrm{where\ } 0 \le I_{i,j}(\theta) \le 1
\end{array}
\end{equation}

where $I_{i,j}(\theta)$ and $K_{i,j}$ are the ($i,j$)th pixels of the image and kernel, respectively. This process is illustrated in Fig.~\ref{fig:avkernels}A.

\subsection{Replication of behavioural experiments}
The equation for describing the bar fixation mechanism shown in Fig.~\ref{fig:bar}C is as follows:
$$
\phi_\mathrm{turn} = \frac{\mathrm{gain}\cdot \pi}{4}\left( \sum\limits_{K\in \mathbf{G}_\mathrm{left}}\max(0,A(I,K,0\degree)) - \sum\limits_{K\in \mathbf{G}_\mathrm{right}}\max(0,A(I,K,0\degree)) \right)
$$
where $I$ is the view of the bar from the agent's current location and $\mathbf{G}_\mathrm{left}$ and $\mathbf{G}_\mathrm{right}$ are the sets of left- and right-hemispheric filters. `Gain' is a parameter to control the gain of the system, and here was set to 2.

For the pattern recognition tasks (see Fig.~\ref{fig:pattern}), the difference in activation is calculated as follows:
$$
D(I) = \sqrt{\frac{\sum\limits_{K\in \mathbf{G}}(A(I,K,0\degree)-A(I,K,90\degree))^2}{|\mathbf{G}|}}
$$
where $\mathbf{G}$ is the set of R2 filters, $I$ is the current pattern pair and $A(\cdot,\cdot,\cdot)$ is the activation of the kernel to the pattern, as described in Equation~\ref{eq:act}.

\subsection{Neural Networks}
The neural networks were executed using the \texttt{Netlab} toolbox for Matlab.
All networks were two-layer feedforward networks, with 10 hidden units and a linear activation function for the output units.
There were 100 training cycles and optimisation was performed with the scaled conjugate gradient method.

\subsubsection{Stimuli}
\label{sec:methods:stimuli}
The stimuli used to train the networks were trained were a series of black `blobs' on a white background.
The blobs were based on ellipses with a fixed ratio between the lengths of the major and minor axes ($2:1$), with the radii modified with complex waves:
$$
r(\theta) \le \left(\frac{\cos^2 \theta}{2} + \frac{\sin^2 \theta}{a} \right)^{-1} + W(\theta), \theta \in \{0, 2\pi\}
$$
where $a$ is the length of the major axis and $W(\theta)$ is a complex wave defined as:
$$
W(\theta) = \sum_{i=1}^n W_i(\theta) = \sum_{i=1}^n A_i \sin f_i (\theta+\phi_i) 
$$
where $A_i$, $f_i$ and $\phi_i$ describe the maximum amplitude, frequency and phase shift of the wave $W_i(\theta)$, respectively.

In these experiments, $A_i$, $f_i$ and $\phi_i$ were randomly generated and $n=2$.
$A_i$ was a random value from 0 to 1, $f_i$ were random integers from 1 to 30 and $\phi_i$ was a random value from 0 to $2\pi$.
\begin{comment}
        nvar = 1000;
        nwave = 2;
        maxfreq = 30;
        maxamp = 1;
\end{comment}

The blobs were first generated, according to the above equation, as an image of $120\times 270$ pixels.
For the `raw view' stimuli, these images were resized, using Matlab's \texttt{imresize} function, to $2\times 14$ pixels, thus giving the same number of inputs as there are R2 filters ($n=28$).

\begin{comment}
\subsubsection*{Grading performance of neural networks}
The performance of neural networks was graded by calculating the \ac{rms} difference between the matrix of true values for the parameters with the network's output:
$$
E(\mathbf{y},\mathbf{t}) = \sqrt{\frac{\sum\limits_{i=1}^{n} (\mathbf{y}_i-\mathbf{t}_i)^2}{n}}
$$
where $E(\mathbf{y},\mathbf{t})$ is the mean error score, computed from the vector of outputs given by the network, $\mathbf{y}$, and the vector of true values, $\mathbf{t}$.
Hence, for a network that computed the values of all parameters accurately, a graph of the network's output \emph{vs.} the true values would give the line $y=x$ and an error score of 0 over the whole range of values.
\end{comment}
