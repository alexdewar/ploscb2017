\begin{figure}
\centering
\includegraphics{figures/methods}
\caption{Diagram showing how data from multiple flies are averaged to produce the prototypical \acp{RF}.
A: The raw image (left; shown is an R4d \ac{RF}, glomerulus 1, fly no. 4) is thresholded so as to give excitatory and inhibitory regions of uniform intensity (right; see Section~\ref{sec:methods:preprocessing}). The `centre' is then calculated as the centroid of the largest excitatory region (+).
B: Example showing aligning for two RFs (glomerulus 1, flies 4 and 5). The new centre is taken as the average of the centre of both \acp{RF}. The \acp{RF} are then shifted so that the centres are all aligned.
C: Averaging of R4d RFs for a single glomerulus (no. 1, flies no. 1--7), following alignment. Note that this is the `left-hand' version; the `right-hand' version is its mirror.}
\label{fig:averagekern}
\end{figure}
