\section{Discussion}
A general problem in neuroscience is understanding how sensory systems organise information to be at the service of behaviour. Computational approaches have always been important in this endeavour, as they allow one to simulate the sensory experience of a behaving animal whilst crucially considering how this information is transformed by populations of neurons. Thus we can relate the details of neural circuitry to theories about the requirements of behaviour.
Work by \citeA{Seelig2013}, showing the forms of visual receptive fields for ring neurons projecting to the ellipsoid body of flies, opens the possibility of investigating how these neurons organise information to be at the service of specific behaviours. These populations of ring neurons are small (15-30 units) and can be thought of as a sensory bottleneck. However, they are essential for certain visually guided behaviours. Here we have taken a computational approach  to investigate the information that makes it through this bottleneck to better understand the behaviours these circuits serve.

\subsection{Are flies performing pattern recognition?}

R2 cells are critical for conditioning in a pattern learning task \cite{Pan2009}. At the synaptic level, expression of \emph{rutabaga} is required \cite{Pan2009}, as is \emph{foraging} upstream of this \cite{Wang2008}. Now that we know the visual characteristics of these cells, we can investigate how pattern discrimination is implemented. Previously, it has been suggested that pattern recognition relies on distinguishing visual patterns on the basis of higher-order properties, such as size, orientation and elevation \cite{Ernst1999,Pan2009}, however it has been found that at the R2 synapses the encoding is independent of any single parameter \cite{Liu2006}. Both of these views are consistent with our analysis. Size and position information are preserved in the R2 population code but do not need to be extracted by specific sub-populations of cells. 

Going one step further we found that the patterns of activity in the R2 population code are a good fit with the learning index of flies for pattern pairs, as shown in \citeA{Ernst1999}. We also found, however, that these cells do not appear to be specifically `pattern recognition' cells.
For example, a great increase in performance is given by simply having more cells or having the RFs more spread out. Any selection pressure on flies ability to discriminate patterns (as bees need to do for instance) would surely have led to a larger R2 population or, more likely, visual input to the Mushroom Body and therefore we can be confident that ring neurons have not been tuned for arbitrary pattern recognition. We therefore suggest caution if research on flies is held up as a possible route to understanding the neural basis of pattern recognition.

Interestingly, flies' spontaneous preference for patterns, which does not involve R2 neurons \cite{Ernst1999}, was not correlated with the values obtained by our simulation. This fits with work showing that flies' preference for novelty involved the ellipsoid body but did not require any one of the R1, R3, R2/R4m or R3/R4d subsets of neurons specifically \cite{Solanki2015}.

\subsection{Short-term memory for object position in flies}

One striking feature of the ring neuron receptive fields is that they are in general tuned to vertically oriented objects.
We also know that fruitflies, like other flies, are strongly attracted to vertical bars which has been leveraged across a range of behavioural paradigms for flies \cite<bar fixation: >{Neuser2008}. In one, single flies are placed into a virtual-reality arena, with two vertical stripes shown 180\degree\ apart.
In this scenario, flies typically head back and forth between the two bars.
Occasionally, however, the bars would disappear when a fly crossed the arena's midline and a new bar appears at 90\degree\ to the old ones.
Flies respond by reorienting to this new target, which then also disappears, whereupon flies will resume their initial heading, even though the original bars are no longer there. This indicates that directional information is stored in short-term memory and updated. Work by \citeA{Neuser2008} has shown that R4 (and R3) ring neurons are involved in a spatial orientation memory for bars, 

Accordingly, we decided to examine the responses of the ring neuron filters to vertical bars and what role they could play in a spatial orientation task.
We found that both R2 and R4d neurons were responsive to vertical bars of varying widths, particularly to the edges of larger bars and the centres of narrower ones, mirroring real flies' behaviour \cite{Osorio1990}.
We also showed with a simulation that the cells would provide sufficient information to guide homing towards a large vertical object, as with a spatial orientation task \cite{Neuser2008}.
We also showed the spatial information for azimuth makes it through the sensory bottleneck and whilst the population code does not perfectly maintain orientation information, it is likely to be a good detector of vertical bars.

The sensory information provided by these cells could be used in a variety of ways and there are suggestions that R4d neurons could form part of a `path integration' system \cite{Neuser2008} or be analogous to mammalian head-direction cells \cite{Tomchik2008}.
Indeed, there is evidence that, as with path integration and head-direction cells, these cells also integrate idiothetic information \cite<e.g.,>{Zars2009}.
This raises the intriguing possibility that ring neurons play a role in a short-term memory system in order to facilitate complex behavioural responses which require integration of multiple sources of information over time, rather than simpler reflexive or classically conditioned behaviours. The fly could be remembering the position of a stimulus, the history of its own movements or both \cite{Tomchik2008}.
Work by \citeA{Guo2015} indicates that R3/R4d neurons, but not R2/R4m, play a role in learned spatial orientation to stimuli other than simple vertical bars.
Flies tethered in a drum were conditioned to fly toward either the left or right of a visual pattern (such as an inverted `T'); it was found that the absence of these neurons prevented conditioning.
This suggests a role for R4d cells in remembering the position of a stimulus with respect to the fly's own movements, as would be required in a path integration system.

\subsection{Summary}

In conclusion, we know that R4d cells provide sufficient information for a bar orientation task and R2 cells for pattern discrimination, although neither of these tasks appear to be the sole function of these sets of cells per se.
This raises the question, what then are the more general roles of these cells for fly behaviour?
Evidence from other sources points to a more multipurpose functionality.
First, the fact that R4d, but \emph{not} R2 cells are involved in a task where flies have to fly towards one or other side of the pattern \cite{Guo2015}, rather than merely avoiding one or another pattern, indicates that they are involved in the encoding of the fly's bearing relative to visual stimuli, perhaps as part of a path integration system \cite{Seelig2015}.
Additionally, the ring neurons are known to be multimodal.
R2 neurons have also been implicated in olfactory behaviours involving a conditioned aversive choice, or an appetitive choice task \cite{Azanchi2013, Zhang2013,Zhang2015}. All this indicates that R2s may be involved in modulating action selection, via multimodal operant conditioning.
This would fit with accounts of the central complex as an action selection system, homologous with the mammalian basal ganglia \cite{Strausfeld2013}.

We feel we have given here not only a novel view on the functions and organisation of the \emph{Drosophila} visual system, but raised issues regarding neural coding in insects more generally. In particular, we would like to challenge the idea that a complex behaviour must be supported by a discrete cognitive module to extract abstract features or properties of stimuli.
\emph{Drosophila}'s limited ability to discriminate patterns using abstract properties seems to be the by-product of a simple visual system tuned to provide information to guide specific behaviours.
In the future, a combined approach -- behavioural research that incorporates insights and predictions from computational models -- could help pave the way to a mechanistic, quantitative account of \emph{Drosophila} behaviour and its relation to sensory information.

