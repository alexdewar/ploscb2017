\begin{figure}[h]
\centering
\includegraphics[width=15cm]{drosopattern/simdiffpatts}
\thesiscontcaption{R2 cells do not encode detailed shape information.}
\label{fig:drosopattern:simdiffpatts}
\end{figure}

\begin{figure}[h]
\contcaption{R2 cells do not encode detailed shape information. A--C: The discriminability of pattern pairs can vary greatly independently of the apparent difference between visual stimuli. A: A pattern pair made up of a triangle and its inverse with both triangles aligned to their lowest points. B: The same triangles, but this time aligned by their centres of mass, make another pattern pair. C: The difference in activation between 0\degree\ and 90\degree\ for all R2 RF filters for the triangles from A (open bars) and B (closed bars). The mean activation difference is greater for the triangles in A over B. The red square marks the output of the ring neuron \ac{RF} shown in A and B. 
D--F: We can also generate shapes that appear similar yet produce a large mean difference in RF activation (D) or appear different and produce similar RF activations (E). The stimuli here are `blobs' of the form described in Methods. An optimisation was performed in \Matlab\ (\texttt{fminsearch} function) to minimise the ratio of blob difference to difference in activation (D) or its inverse (E). Pairs of stimuli are shown in grey and green whereas in the simulation both are black.
F: The corresponding activations of separate (left-hemispheric) R2 glomeruli are shown at the bottom. Two similar patterns give a mean difference in activity of 11.0\%. Two very different patterns give a mean difference in activity of 5.10\%.
}
\end{figure}
