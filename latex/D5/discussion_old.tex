\section*{Discussion}
The ring neurons of the ellipsoid body are important for visually guided behaviours.
Given descriptions of the \acp{RF} of these cells we have related the information encoded in the small sub-populations of ring neurons to the requirements of known behaviours.
In this paper, we have tried to show that a range of behavioural experiments on fruitflies can be matched by modelling known receptive field properties of \emph{Drosophila} ring neurons as visual filters.

[From presub: A general problem in neuroscience is understanding how sensory systems organise information to be at the service of behaviour. Computational approaches have always been important in this endeavour, as they allow one to simulate the sensory experience of a behaving animal whilst crucially considering how this information is transformed by populations of neurons. Thus we can relate the details of neural circuitry to theories about the requirements of behaviour. In visual neuroscience, a rare opportunity has emerged to understand how visually guided behaviours are served by visual encoding. Small populations of identifiable neurons in Drosophila are known to be necessary for particular visual tasks and the individual receptive fields of the neurons in these populations have now been described entirely. In our manuscript we have taken a computational approach to consider how the population code of these neurons relates to the information required to control specific behaviours in Drosophila.]

First, we looked at the responses of our R2 and R4d filters to vertical bars, to which \emph{Drosophila}, like other flies \cite{Reichardt1969}, show a strong response.
While the R2 filters were maximally responsive to the inside edges of large bars, the R4d filters responded most at headings $\pm90\degree$ from the centres.
Neuser and colleagues \cite{Neuser2008} showed that R4 neurons play a role in a spatial orientation memory for bars, with a modified version of Buridan's paradigm, in which flies can be seen to walk back and forth between two vertical bars 180\degree\ apart until exhaustion \cite{Bulthoff1982}.
R2 neurons, however, have not yet been shown to play a role.
While we are not suggesting that the R4 cells play a role in the initial orientation towards a bar, it is clear that they form the basis for the encoding of the memory \cite{Guo2015,Neuser2008}.
[Emphasise that information provided by cells is perfect for behaviour. Strong about positives!]

[Need more general intro leading to 2 or 3 Qs. \\
(1) What is the natural behaviour being studied? \\
(2) What do we know about the wiring? \\
Echoing the extensive research with bees, research suggests flies also have pattern-like abilities, for which R2 cells are essential.
This raises the Q of whether the things we found...]
We then turned to pattern recognition and found that patterns that were discriminable by flies in \cite{Ernst1999} were also discriminable in our simulation.
However, we only examined spontaneous pattern preference, whereas Ernst and Heisenberg \cite{Ernst1999} also looked at whether pattern preferences could be learned.
This included pattern pairs with and without a spontaneous preference and with and without learning possible, in all combinations.
Liu and colleagues \cite{Liu2006} have shown that the learning is dependent on plasticity at R2 synapses, but how spontaneous preferences arise (whether the preference can be modified by learning or not) is still not known.
Although many of Liu et al.'s \cite{Liu2006} flies did show spontaneous preferences, this was only on an individual basis and was not maintained by flies lacking the \emph{rutabega} gene, which is essential for learning; patterns which elicited a spontaneous preference on a population level were not used.
Hence, these individual preferences probably reflect individual differences in flies' perceptual systems, whereas the origin of the interaction between how spontaneous preference and learnability of patterns in Ernst and Heisenberg \cite{Ernst1999} is still not known.

The other question is what form the encoding of these patterns takes.
Earlier work \cite{Dill1993}, showing that learned pattern preferences disappear if the patterns are elevated by 9\degree, suggested that the patterns are retinotopically matched, a model that has also been used to account for insect navigation \cite{Zeil2003,Cartwright1983}.
However, in more recent work \cite{Pan2009,Liu2006,Ernst1999}, it has been argued that visual patterns are encoded in terms of parameters such as size, elevation, contour orientation and vertical compactness.
In \cite{Pan2009}, although it is shown that explicit encoding of two of the visual parameters can be found in the fan-shaped body, R2 and R4m neurons were found to encode visual stimuli in a parameter-independent manner.
Although we argue that to perform pattern discrimination ring neuron receptive fields by themselves contain enough information (see above), we were interested in what information passes through the bottleneck given by the small number of ring neurons, possibly to be encoded at the level of the central complex.
We found that orientation, size and elevation all pass through this bottleneck, without explicit encoding.
Contour orientation, however, was the weakest-performing parameter, mirroring what Guo and colleagues \cite{Guo2015} recently found in a similar pattern discrimination paradigm to the one discussed previously.

One problem with interpreting the pattern recognition experiments that have been performed in \emph{Drosophila} is that it is not clear exactly what `natural' behaviour is being examined:
Do fruitflies need to be able to learn arbitrary visual stimuli?
While a honeybee, for example, has to learn arbitrary shapes corresponding to different flowers, flies face no comparable ecological challenge.
Hence, although R2 neurons have been implicated in these tasks, it is possible that this may not tell us much about their primary function in wild \emph{Drosophila}.
[Indeed, pattern discrimination/coding can be improved...]

[In the future, improved knowledge about the connectivity and response patterns of ring neurons, including and besides the R2 and R4d subsets, will make stronger models of \emph{Drosophila} behaviour possible.
Among other things, ring neurons are multimodal and in some cases possibly connected to motor areas, and these aspects could be incorporated.
We would also like to see behavioural assays involving stimuli informed by these receptive field properties, which would show if and where predictions made by models fail.
Finally, greater knowledge about \emph{Drosophila} visual ecology would also help inform both experiments and interpretations of understood properties of visual cells.] [Move elsewhere in text.]

[From presub: This work presents novel insights into the visual organisation of Drosophila but also has implications for neural coding in other insects and challenges the notion that there are specific modules for extracting abstract visual parameters. Rather, the limited ability to discriminated patterns using abstract properties seems to be a by-product arising from a simple visual system tuned to provide information to guide specific behaviours. We feel that the general interest in visual coding and in Drosophila neuroscience means that this paper is of sufficient general interest for your readership.]