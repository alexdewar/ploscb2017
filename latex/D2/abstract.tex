\begin{abstract}
All organisms wishing to survive and reproduce must be able to respond adaptively to a complex, changing world.
Yet the computational power available is constrained by biology and evolution, thus favouring mechanisms that are parsimonious yet robust.
Here we show the power given by a small number of visually responsive neurons in \emph{Drosophila melanogaster} for solving complex behavioural tasks including pattern recognition.
Our findings indicate that flies' performance on some tasks may be better understood by taking a bottom-up approach and looking at the combined activity of general processing mechanisms, rather than as the product of more specialised cognitive modules.

\end{abstract}
