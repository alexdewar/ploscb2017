\section{Introduction}

Insects, like vertebrates, exhibit a great many behaviours key to survival, including place-homing \cite{Ofstad2011}, recognition of individual conspecifics \cite{Sheehan2008}, flight control \cite{Taylor2001,Gotz1987}, complex communication [e.g. von Frisch] and nest-building [e.g. termite stuff].
Moreover, these feats are carried out with nervous systems orders of magnitude smaller than their vertebrate cousins'.

Questions of insect cognition are tackled broadly in one of two ways by researchers: either a `top-down' or a `bottom-up' approach.
A top-down approach is to look for known vertebrate behaviours in insects.
A bottom-up approach, by contrast, seeks to examine how insects' behaviours are tailored to specific, ecologically relevant tasks.
For example, male fiddler crabs \emph{Uca pugilator} has two visual tasks that it must perform to survive and reproduce.
One is to avoid predators, the other is to attract females or repel other males, both of which are achieved by a claw-waving display.
As it turns out, these tasks can be carried out with a simple heuristic: treat anything above the horizon as a predator and anything below as a conspecific \cite{Layne1997}.
In this way, what at the outset one might think of as two `visual recognition' tasks can be greatly simplified, along with the neural machinery needed to support it.

A common criticism of this kind of approach is that it is reductionist and oversimplifies animal behaviour [see Dennett, and critique in \cite{Shettleworth2010}].
Certainly, we are a long way from a total understanding of even moderately complex behaviours, however `simple' the model organism used.
%However, to reject more parsimonious models of animal behaviour, where they are backed by behavioural data, simply on the grounds that it is doing a `disservice' to the an
It is arguably just as much a `disservice' to an animal's cognitive abilities to measure them against human beings', a situation in which by definition they will come up short \cite{Doring2011}.

Sometimes, this approach can shed new light on conventional animal experiments.
For example, \cite{Wystrach2011} showed that ants (\emph{Gigantiops destructor}) make the same kinds of error as vertebrates on a task taking to indicate that animals code their spatial surroundings in terms of geometry.
An animal is trained to find food in one corner of a rectangular arena; when the animal is released during testing, it will head for either the correct corner or its diagonal opposite with equal likelihood.
Moreover, this holds true even when the two corners are disambiguated by the presence of additional visual features, and this is taken to indicate that there is a geometry module which takes precedence over other visual stimuli in a spatial task.
However, \cite{Wystrach2011} showed that, in the case of their ants, the data were better explained with a simple image-matching process (rotational image difference function) \cite{Philippides2011,Zeil2003} than by a model based on visual geometry.

\begin{comment}
Faced with this startling fact, there are two common reactions.
The first is to denigrate insects' intelligence: to argue that what appear to be complex behaviours are really just the combined actions of elaborate reflexes, a view shared by Aristotle and Descartes.
The second is to defend insects, by offering up the range and diversity of insect behaviour as evidence that invertebrates are sometimes as `clever' as vertebrates.
Our own viewpoint differs from both of these.
We instead argue that while insects possess an impressive behavioural repertoire, the fact that these behaviours may be partially reducible to simple, task-specific competencies is precisely what makes them such a fruitful model for computational neuroscience.

Vision is known to play a role in a many behaviours exhibited by the fruit fly \emph{Drosophila melanogaster} \cite<review:>{Borst2014}, including fixation towards vertical bars [cits], avoidance of expanding objects in the lateral visual field [cits], landing [name+cits in house fly 53--57], escape \cite{Card2008}, the optomotor response [cits] and mate pursuit [name+cits] [\& others?].
The extent to which these behaviours depend on specialised, task-specific modules \emph{vs.} common, more `general purpose' machinery is currently not known.
What might be expected, however, given that fruit flies possess only a few hundred thousand neurons \cite{Borst2014}, is that evolution will have favoured, where possible, a single computational unit supporting multiple behaviours over multiple, hyperspecialised units performing the same function.

[For example, the landing and avoidance responses are both triggered by expanding visual stimuli and are both implemented in the lobula plate -- expand and check.]

\end{comment}

Several classes of ring neuron in the \emph{Drosophila} ellipsoid body have been implicated in visually guided behaviours (R1: place homing \cite{Ofstad2011,Sitaraman2010,Sitaraman2008}; R2: pattern recognition \cite{Pan2009,Liu2006,Ernst1999}; R4: bar fixation \cite{Neuser2008}).
Recently, \cite{Seelig2013} have described the form of the visual \acp{RF} for two subtypes of ring neuron---R2 and R4d---of which there are only 14 and 28, respectively.
In this paper, we intend to show that behavioural responses to experimental stimuli known to depend on these neurons are predicted on the basis of the information implicit in these neurons' responses, without the need for this information to be fed into specialised cognitive mechanisms.

