\section*{Discussion}
In this paper, we have tried to show that a range of behavioural experiments on fruitflies can be matched with simple simulations using the known receptive field properties of \emph{Drosophila} ring neurons as visual filters.
Firstly, we looked at the complementary roles played by R2 and R4 neurons in bar fixation.
It is known that while R2 neurons are required for a fly to aim toward a vertical bar, R4 neurons are required for saccading to bars roughly 90\degree from the centre of the visual field.
These separate functions are predicted by the different responses to R2 and R4 cells to vertical bars (Figure~\ref{fig:bars}): Whereas R2 neurons respond maximally to the centres of narrow bars and the inside edges of wide bars, R4 neurons respond instead to bars at approximately 90\degree.
A simple mechanism connecting these neurons to motor output could then be used to guide fixation towards a bar.

Secondly, we investigated computationally how R2 neurons are involved in pattern recognition \cite{Ernst1999,Pan2009,Liu2006}.
Our simulation was able to discriminate the patterns discriminable by flies and unable to discriminate those the flies failed on.

The question remains, however, as to how behaviourally relevant these experiments are.
For one thing, fruitflies are not specialised for discriminating arbitrary visual patterns and it difficult to imagine the selection pressure that could have led to an ability like this, in contrast to, say, honeybees which learn the shapes of individual flowers [do they?].
We should not suppose that just because fruitflies can perform a pattern recognition task this reflects the functioning of a specialised capacity; rather it could be based on more generic features of the \emph{Drosophila} nervous system.
For example, it is well established that stimuli passed through a nonlinear filter makes discrimination easier.
The choice of filter can be fairly arbitrary: Fernando and Sojakka \cite{Fernando2003} used motors causing ripples in a bucket of water as a way of `processing' speech stimuli, which were highly discriminable by a perceptron trained with images recorded with a camera above the bucket.

We know that the pattern learning takes place at the synapses of R2 neurons onto the ellipsoid body [check] \cite{Pan2009}.
It could be that a similar process is taking place here, with the \acp{RF} acting as nonlinear filters, with the `weights' of each output altered by plasticity so as to modify the spontaneous preference.

There is an underlying assumption that in order to distinguish two shapes, an agent must first encode one or more parameters such as size, vertical centre of mass, etc., then compare the score for each shape.
Alternatively, pattern discrimination can take place much more simply with a comparison of relatively `raw,' unprocessed information.
The latter is often more effective in addition to being more parsimonious [cit].
In the case of the R2 neurons, this gives one dimension per neuron, so 28 along which a stimulus can vary.
This alone gives ample information for discriminating any number of stimuli, although of course, how successful the agent will be at learning a pair of stimuli will depend on factors such as the size of the \acp{RF}, where they are on the retina, etc.
However, why a given pair of stimuli is discriminable is unlikely to be obvious to a human observer.
A real understanding can only come by looking at the outputs of individual R2 neurons.
The R2 neurons in of themselves constitute a kind of visual `encoding,' although the ways in which the visual information are encoded are not likely to lend themselves easily to a good `label.'

Similarly, we were able to discriminate \emph{all} of the different patterns in \cite{Ernst1999} with a perceptron, though, obviously, the flies cannot.
If the correct amount of noise were applied to the outputs of the R2 cells in our simulation, our results would likely be comparable to the behavioural data.

%Finally, we showed that a two-layer neural network can be trained to extract the 
