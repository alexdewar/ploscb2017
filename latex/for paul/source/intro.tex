\section{Introduction}
% [PRG]
% incorporating andy's comments 26/09/15

As with many animals, vision plays a key role in a number of behaviours performed by the fruitfly \emph{Drosophila melanogaster}, including mate-recognition \cite{Agrawal2014}, place homing \cite{Ofstad2011}, visual course control \cite{Borst2014}, collision-avoidance \cite{Tammero2002}, landing \cite{Tammero2002} and escaping a looming object (like a rolled newspaper, for example) \cite{Card2008}. The benefit of studying these visually guided behaviours in \emph{Drosophila} is, of course, the range of neurogenetic techniques which give a realistic chance of understanding the neural circuits that underpin them. With that goal in mind, we focus on recent work \cite{Seelig2013} which has mapped the receptive fields of a set of visually responsive neurons, the ring neurons of the ellipsoid body, which are surprisingly small in number given they are key for certain complex behaviours \cite{Seelig2015,Neuser2008,Liu2006}. To understand their role in these behaviours, it is desirable to investigate the response of these cells for experiments which elucidate its behaviour. In this chapter, we use modelling to bridge this gap between neurogenetic data and behaviour by evaluating neural responses during simulated behaviour. In this way we investigate how small populations of well-described visual neurons in \emph{Drosophila} provide behaviourally relevant information.

In laboratory assays flies show interesting spontaneous visual behaviours. For instance, flies will orient towards bar stimuli \cite{Reichardt1969,Gotz1987} and in a circular arena with two diametrically placed bars will walk between them for long periods. This spontaneous preference for elongated vertical bars is reduced as the bar is shortened until free flying flies show a spontaneous aversion to small cube stimuli \cite{Maimon2008}. In addition to a suite of visual reflexes, \emph{Drosophila} also demonstrate complex visual behaviours involving interactions between orientation and memory.
For instance, a number of papers have investigated the process of pattern recognition and its neural underpinnings \cite{Pan2009,Liu2006,Ernst1999}.
The standard paradigm involves putting a fly into a closed-loop system where it is tethered in a drum, on the inside of which are two visual stimuli alternating every 90\degree\ (Figure~\ref{fig:recap}E). As the fly attempts to rotate in one direction, the drum counter rotates, giving the illusion of closed-loop control. To elicit conditioned behaviour, if the fly faces one of the patterns it receives negative reinforcement, a heat beam or laser which heats the abdomen. Over time if the fly is able to differentiate the patterns it will preferentially face the unpunished pattern. This procedure has been used to demonstrate that flies can differentiate stimulus pairs such as upright and inverted `T' shapes, a small and a large square, and many others \cite{Ernst1999}. That is, flies seem to possess a form of pattern recognition and pattern memory analogous to the better studied pattern memory of bees \cite{vonFrisch1914,Giurfa1997,Horridge2009}.

Here we re-examine these experiments by simulating the visual input as it would be processed through newly-mapped visually responsive cells. The control of these visual behaviours is dependent on the central complex of flies, a brain area thought to be involved primarily in spatial representation and mediation between visual input and motor output \cite{Pfeiffer2014}.
The central complex comprises the ellipsoid body, the fan-shaped body, the paired noduli and the protocerebral bridge \cite{Young2010}.
This part of the brain has been characterised as the site of action selection and organisation and is claimed to be homologous to the basal ganglia in vertebrates \cite{Strausfeld2013}.
In the ellipsoid body, there are a class of neurons called `ring neurons', which are known to be involved in visual behaviours (R1: place homing \cite{Ofstad2011,Sitaraman2010,Sitaraman2008}; R2/R4m: pattern recognition \cite{Pan2009,Liu2006,Ernst1999}; R3/R4: bar fixation memory \cite{Neuser2008}).

Beyond the identification of brain regions associated with specific behaviours it is now possible to describe the properties of specific visual cells in the central complex. Seelig and Jayaraman \cite{Seelig2013} have studied two classes of ring neuron in the \emph{Drosophila} ellipsoid body.
The two subtypes of ring neuron investigated were the R2 and R4d ring neurons, of which only 28 and 14, respectively, were responsive to visual stimuli.
The cells were found to possess \acp{RF} that were large, centred in the ipsilateral portion of the visual field and with forms similar to those of mammalian simple cells \cite{Hubel1962}.
Like simple cells, many of these neurons showed strong orientation tuning and some were directionally sensitive.
The ring neuron \acp{RF}, however, are much coarser in form than simple cells, are far larger, are less evenly distributed across the visual field and respond mainly to orientations near the vertical.
This suggests that ring neurons might have a less general function than simple cells \cite{Wystrach2014}.
The population of simple cells means that small high contrast boundaries of any orientation are detected at all points in the visual field.
Thus the encoding provided by simple cells preserves visual information about bars and edges that could be used as a `general-purpose' network feeding into any number of behaviours.
In contrast, the coarseness of the receptive fields of ring neurons, allied to the tight relationship between specific behaviours and sub-populations of ring neurons suggests instead that these cells are providing economical visual information in a behaviourally tuned way.

To investigate such issues, we here advocate the use of a synthetic approach whereby investigations, in simulation, of the information provided by these populations of neurons can be related to behavioural requirements, thus `closing the loop' between brain and behaviour. We show how the population code is well-suited to the spontaneous bar orientation behaviours shown by flies. Similarly, we verify that our population of simulated ring neurons are able to discriminate visual patterns to the same standard as flies.
Upon deeper analysis, we demonstrate that certain shape parameters -- orientation, size and position -- are implicit in the ring neurons' outputs to a high accuracy, thus providing the information required for a suite of basic fly behaviours.
This contrasts with the rather limited ability of ring neuron populations (and flies) to discriminate pattern pairs, casting doubt on more cognitive explanations of fly behaviour in pattern discrimination assays.
