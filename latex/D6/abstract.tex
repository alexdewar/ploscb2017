\section*{Abstract}
All organisms wishing to survive and reproduce must be able to respond adaptively to a complex, changing world.
Yet the computational power available is constrained by biology and evolution, thus favouring mechanisms that are parsimonious yet robust.
Here we investigate the information carried in small populations of visually responsive neurons in \emph{Drosophila melanogaster} .
These small populations of so-called `ring neurons', found in the ellipsoid body of the central complex, are key for solving complex behavioural tasks including pattern recognition.
Recently the receptive fields of these neurons have been mapped, allowing us to take a bottom-up approach in investigating what behaviours would be supported by these small population codes.
In this instance, computational modelling is key in investigating how perceptual circuits put information at the service of behaviour rather than how they preserve visual information.