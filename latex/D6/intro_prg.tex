\section*{Introduction}
As with many animals, vision plays a key role in a number of behaviours performed by the fruitfly \emph{Drosophila melanogaster}, including mate-recognition \cite{Agrawal2014}, place homing \cite{Ofstad2011}, visual course control \cite{Borst2014}, collision-avoidance \cite{Tammero2002}, landing \cite{Tammero2002} and escaping a looming object (like a rolled newspaper, for example) \cite{Card2008}.
Of course \emph{Drosophila} are unique as the great range of neurogenetic techniques that are available give a realistic chance of understanding the neural circuits that underpin these visually guided behaviours. With that goal in mind, we investigate how small populations of well-described visual neurons in \emph{Drosophila} provide behaviourally relevant information.

In addition to a suite of visual reflexes, \emph{Drosophila} also demonstrate complex visual behaviours involving interactions between orientation and memory.
For instance, there have been a number of papers investigating the process of pattern recognition and its neural underpinnings\cite{Pan2009,Liu2006,Ernst1999}.
The standard paradigm involves putting a fly into a closed-loop system where it is tethered in a drum, on the inside of which are two visual stimuli alternating every 90\degree HERE PUT A REFERENCE TO OUR FIGURE. As the fly attempts to rotate in one direction, the drum counter rotates, giving the illusion of closed-loop control. If the fly faces one of the patterns it receives negative reinforcement, a laser which heats the abdomen. Therefore over time if the fly is able to differentiate the patterns it will preferentially face the unpunished pattern. This procedure has been used to demonstrate that flies can differentiate stimuli such as upright and inverted triangles, upright and inverted `T' shapes, a small and a large square, etc. \cite{Ernst1999}. That is, flies seem to possess a form of pattern recognition and pattern memory analogous to the better studied pattern memory of bees [HERE PUT 2/3 REFERENCES].

Another classic behavioural paradigm for flies (bar fixation \cite{Neuser2008}) has been used to highlight other interesting visually guided behaviours.
Single flies are placed into a virtual-reality arena, with two vertical stripes shown 180\degree\ apart.
In this scenario, flies typically head back and forth between the two bars.
Occasionally, however, the bars would disappear when a fly crossed the arena's midline and a new bar appears at 90\degree\ to the old ones.
Flies respond by reorienting to this new target, which then also disappears, whereupon flies will resume their initial heading, even though the original bars are still invisible. This indicates that directional information is stored in short-term memory and updated.

The control of these visual behaviours is dependent on the central complex of flies, a brain area thought to be involved primarily in spatial representation and mediation between visual input and motor output \cite{Pfeiffer2014}.
The central complex comprises the ellipsoid body, the fan-shaped body, the paired noduli and the protocerebral bridge \cite{Young2010}.
This part of the brain has been characterised as the site of action selection and organisation and is claimed to be analogous/homologous to the basal ganglia in vertebrates \cite{Strausfeld2013}.
In the ellipsoid body, there are a class of neurons called `ring neurons', which are known to be involved in visual behaviours (R1: place homing \cite{Ofstad2011,Sitaraman2010,Sitaraman2008}; R2/R4m: pattern recognition \cite{Pan2009,Liu2006,Ernst1999}; R3/R4: bar fixation memory \cite{Neuser2008}).

Beyond the identification of brain regions associated with specific behaviours it is now possible to describe the properties of specific visual cells in the central complex. Seelig and Jayaraman \cite{Seelig2013} have studied two classes of ring neuron in the \emph{Drosophila} ellipsoid body.
The two subtypes of ring neuron investigated were the R2 and R4d ring neurons, of which only 28 and 14, respectively, were responsive to visual stimuli.
The cells were found to possess \acp{RF} that were large, centred in the ipsilateral portion of the visual field and with forms similar to those of mammalian simple cells \cite{Hubel1962}.
Like simple cells, many of these neurons showed strong orientation tuning and some were directionally sensitive.
The ring neuron \acp{RF}, however, are much coarser in form than simple cells, are far larger, are less evenly distributed across the visual field and respond mainly to orientations near the vertical.
This suggests that ring neurons might have a less generalistic function than simple cells. The population of simple cells ensures small high contrst boundaries of any orientation are detected at all points in the visual field. Thus the encoding provided by simple cells preserves visual information. Whereas the coarseness of the receptive fields of ring neurons, allied to the tight relationship between specific behaviours and sub-populations of ring neurons suggests that these cells are providing economical visual information in a behaviourally tuned way.

Here we advocate the use of a synthetic approach whereby investigations, in simulation, of the information provided by these populations of neurons can be related to behavioural requirements. We verify that our population of simulated ring neurons are able to discriminate the visual patterns to the same standard as flies.
However, although ring neuron-like cells can be used for a limited ability distinguish stimuli differing along some arbitrary pattern dimensions, this doesn't mean this is the natural function of these neuronsCertain parameters---`orientation', `size' and `elevation'---are implicit in the ring neurons' outputs to a high accuracy. Thus providing the information required for basic fly behaviour and casting doubt on more cognitive explanations of fly behaviour in pattern discrimination assays.
 


SOME TEXT THAT WOULD BE BETTER IN THE DISCUSSION
Most of the ellipsoid body's connections, incoming and outgoing, are to the lateral accessory lobes and lateral triangle, though unfortunately little is known about more downstream targets \cite{Pfeiffer2014,Young2010}.


SOME TEXT THAT WOULD BE BETTER IN THE RESULTS
Short-term memory for visually defined headings has been shown to be dependent on the expression of a molecule involved in memory formation (S6 kinase II) in R3 and R4 ring neurons.
Mutant flies ($ign^{\emph{58/1}}$) without this expression reoriented towards the new bar in the bar fixation paradigm, but when it disappeared they did not turn to face their original direction.
When expression was restored in these flies, they would again reorient towards the (invisible) original bar location.

