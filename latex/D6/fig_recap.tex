\begin{figure}
\centering
\includegraphics{figures/recap}
\caption{Simulation of fly pattern discrimination paradigm. A standard experimental paradigm for testing the pattern discrimination abilities of \emph{Drosophila} \protect\cite{Ernst1999} can be replicated in simulation.
A: A diagram showing Buridan's paradigm \protect\cite{Bulthoff1982,Gotz1980}. If a fly is placed in an arena between two large vertical bars, it will walk back and forth until exhaustion.
B and C: The mean output of R2 (blue) and R4d (green) filters, from different headings, to bars of different widths. The blue crosses in A indicate the points from which bars B and C are being viewed.
D: The fly is held tethered in a drum. As the fly attempts to rotate about its yaw-axis, the drum rotates in the opposite direction, thus allowing the fly to select the portion of the pattern in view.
By monitoring the fly's heading, one can surmise whether there is a spontaneous preference for one of the patterns.
Whether the fly can learn to head towards one pattern is tested by adding a laser that punishes the fly for facing one of the patterns.
Shown inside the drum are the visual \acp{RF} for one pair of left- and right-hemispheric glomeruli.
E and F: The \ac{rms} difference in output for R2 (blue) and R4d (green) neurons as the pattern is rotated.
The reference activities are the \ac{RF} outputs when the simulated flies are at 0\degree.
Patterns with a greater difference in activity at 0\degree\ \emph{vs} 90\degree\ should be more discriminable by flies.
For two pairs of patterns we show that there is a much smaller difference in output when the triangles are aligned about the vertical centre of mass (E) than not (F).
This mirrors real flies' performance on this task \protect\cite{Ernst1999}.}
\label{fig:recap}
\end{figure}
